\documentclass[12pt]{article}
\usepackage[utf8]{inputenc}
\parindent 0px
\usepackage{graphicx}
\usepackage{amsfonts,amssymb,amsmath}
\usepackage{commath}
\usepackage{mathtools}
\title{ASSIGNMENT1}
\author{Gaurav Kumar Gautam-sm21mtech12013}\\
\date{August 2021}

\begin{document}

\maketitle
\section*{Chapter II, Examples II}


\textbf{ Question22(iv):- Find the condition that the four points $(x_1,y_1),(x_2,y_2)(x_3,y_3),(x_4,y_4)$ may be the vertices of a parallelogram.}\\[6pt]
\textbf{Sol}:- (i) Midpoints of diagonals of parallelogram are equal.\\[6pt]
 (ii) Opposites sides of parallelogram will be equal and parallel.
 \begin{center}
     \includegraphics[width=.6\textwidth]{Parallelogram.png}
 \end{center}
\In the above diagram \myvec{AC} and \myvec{BD} are diagonals.\\
Mid point of diagonals will be equal.
$$\frac{(\vec{x_1}+\vec{x_3})}{2},\frac{(\vec{y_1}+\vec{y_3})}{2}=\frac{(\vec{x_2}+\vec{x_4})}{2},\frac{(\vec{y_2}+\vec{y_4})}{2}$$\\
Opposite side \myvec{AB} and \myvec{CD} or \myvec{AD} and \myvec{BC}  wil be equal.\\
$$\sqrt{(\vec{x_2}-\vec{x_1})^2 + (\vec{y_2}-\vec{y_1})^2}=\sqrt{(\vec{x_4}-\vec{x_3})^2 + (\vec{y_4}-\vec{y_3})^2}$$\\
$$\sqrt{(\vec{x_4}-\vec{x_1}^2) + (\vec{y_4}-\vec{y_1})^2}=\sqrt{(\vec{x_3}-\vec{x_2})^2+(\vec{y_3}-\vec{y_2})^2}$$\\
If all these condition of any vertices follows then we can say this vertices is for parallelogram.

\end{document}
